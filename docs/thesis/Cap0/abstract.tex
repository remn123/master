The capability of handling unsteady flows over complex geometries with efficiency and high-order accuracy is quite often desirable for the aerospace industry. Recent research on high-order overset flux reconstruction methods have shown successful results over moving boundary problems without the need of remeshing. The ability of accurately handling such type of requirements is very important, for instance, when addressing rotary wing flows and similar problems. In the present study, a high-order flux reconstruction solver is implemented coupled with a high-order conservative interpolation approach and a kd-tree algorithm over the mesh nodes for data communication in the overset regions. The flow is modeled by the 2-D Euler equations discretized in space with a Spectral Differences method and an explicit strong stability-preserving Runge-Kutta scheme for time integration. In the overset grid scenarios, two unstructured grids are generated: a background mesh including all the fluid box domain and a near-body mesh. In order to determine which cell of the donor grid embeds a specific node in the receiver grid, both mesh nodes are implemented with a tree data structure expecting the logarithmic time complexity of O(k.logN) in the geometric search, where k is the number of flux points over the boundary interfaces of the receiver mesh and N the number of nodes in the donor grid. Furthermore, the solution is interpolated from the donor cell to the receiver grid point based on the donor cell polynomial expansion at the solution points and, then, imposed in a weak manner as a boundary condition to exactly reconstruct the flux with the approximate Riemann solver, as for any other interior face. The implementation is tested for different validation problems from the 5th International Workshop on High-Order CFD Methods (HiOCFD5). Additionally, accuracy and convergence studies are performed on both single and overset grid approaches and compared to results in the literature.