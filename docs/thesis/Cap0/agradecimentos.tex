Antes de tudo, eu gostaria de profundamente agradecer a minha mãe, Antonia, por todo amor, confiança e ensinamentos que me proveu. É graças ao seu incondicional amor e esforço que pude chegar aqui. Ao seu lado, gostaria também de agradecer por todo o suporte e carinho do meu padrasto, José Eraldo, que sempre me ajudou em tudo o que eu precisei. Em especial, eu gostaria de agradecer a toda força, paciência e incentivo que minha amada esposa, Ana Paula, tem me dado durante todo esse tempo. Tenho um extremo orgulho da nossa história que sem sombra de dúvidas me fez e ainda me faz uma pessoa melhor. 

Gostaria de agradecer ao meu orientador, Professor Azevedo, não somente pelo massivo conhecimento que possui, mas primordialmente pela sua ética e profissionalismo. Acredito que de todas as oportunidades que tive dentre conversas e aulas, não houve uma única vez em que não sai carregado de ânimo a aprender e extremamente motivado. Próximo ao Lab, gostaria de especialmente agradecer ao Fábio Mallaco, Pepê, Leo e Marco por toda ajuda e ensinamentos. 

Um pouco mais longe do ITA, gostaria de agradecer aos meus amigos de vida: André, Sabiá e Daniel Thoma. Sem eles o tempo de ensino médio, faculdade, e mestrado, não teria sido tão empolgante e divertido. Em particular, gostaria de agradecer o André pelas diversas conversas técnicas que trocamos, por todas as vezes que me ajudou a entender melhor sobre um problema e por toda a proximidade que temos até hoje. Muito obrigado por tudo, amigos.

Gostaria de declarar meus sinceros agradecimentos à minha segunda família: Willians, Valquíria, Fernando, Amanda, inclusive o Guilherme. Por pelo menos metade da minha vida, essas pessoas têm me proporcionado grandiosíssimos momentos de felicidade e memoróveis confraternizações. Por fim, ao Marquinhos que será cena de um futuro capítulo.