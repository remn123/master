A capacidade de lidar com escoamentos n{\~a}o-estacion{\'a}rios sob geometrias complexas de forma eficiente e com alta ordem de acur{\'a}cia {\'e} frequentemente bastante desej{\'a}vel pela ind{\'u}stria aeroespacial. Pesquisas recentes no contexto de malhas sobrepostas, que promovem alta ordem no processo de reconstru\c{c}{\~a}o dos termos de fluxo das equa\c{c}{\~o}es da mec{\^a}nica dos fluidos, t{\^e}m apresentado {\'o}timos resultados para aplica\c{c}{\~o}es que incluem corpos m{\'o}veis sem a necessidade de novas gera\c{c}{\~o}es de malha. A habilidade de propostas que acoplam tais requisitos {\'e} de extrema import{\^a}ncia, por exemplo, ao se considerar escoamentos que envolvam asas rotativas ou problemas similares. No presente trabalho, um m{\'e}todo num{\'e}rico de alta ordem de precis{\~a}o no processo de reconstru\c{c}{\~a}o dos termos de fluxo {\'e} implementado e adaptado para considerar t{\'e}cnicas de interpola\c{c}{\~a}o conservativa de alta ordem com uma estrutura de dados sobre os n{\'o}s da malha via o algoritmo kd-tree para efetuar a comunica\c{c}{\~a}o de informa\c{c}{\~a}o entre malhas sobrepostas. O escoamento {\'e} modelado pelas equa\c{c}{\~o}es de Euler em 2-D discretizadas no espa\c{c}o pelo m{\'e}todo das Diferen\c{c}as Espectrais (SD) e um m{\'e}todo expl{\'i}cito de integra\c{c}{\~a}o temporal via um m{\'e}todo SSP Runge-Kutta. Em cen{\'a}rios de malhas sobrepostas, duas malhas n{\~a}o estruturadas s{\~a}o geradas: uma malha de fundo incluindo todo a caixa de dom{\'i}nio do fluido e uma malha pr{\'o}xima ao corpo de interesse. A fim de determinar qual c{\'e}lula da malha doadora envolve um n{\'o} espec{\'i}fico da malha receptora, ambas as malhas s{\~a}o representadas por uma estrutura de dados em {\'a}rvore de seus n{\'o}s, a qual prov{\^e} uma complexidade temporal logar{\'i}tmica de ordem O(k.logN) para o processo de busca geom{\'e}trica, onde k {\'e} o n{\'u}mero de pontos de fluxo alocados ao longo do contorno externo da malha receptora e N {\'e} o n{\'u}mero de n{\'o}s da malha doadora. Al{\'e}m disso, a solu\c{c}{\~a}o {\'e} interpolada das c{\'e}lulas doadoras para os pontos na malha receptora atrav{\'e}s da expans{\~a}o polinomial pelos pontos de solu\c{c}{\~a}o da c{\'e}lula doadora e, em seguida, imp{\~o}e-se como condi\c{c}{\~a}o de contorno  fraca para reconstruir o fluxo de forma exata via um esquema de solu\c{c}{\~a}o aproximada do problema de Riemann tal qual {\'e} feito em outras faces entre c{\'e}lulas. A implementa\c{c}{\~a}o {\'e} avaliada em diferentes testes de valida\c{c}{\~a}o extra{\'i}dos do evento internacional de m{\'e}todos de alta ordem em CFD - 5th International Workshop on High-Order CFD Methods (HiOCFD5). Por fim, estudos da acur{\'a}cia e converg{\^e}ncia s{\~a}o realizados para ambos os casos de malha {\'u}nica e malhas sobrepostas comparando-se os resultados com a literatura.