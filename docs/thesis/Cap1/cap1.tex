\section{Motivation of the Present Work}
Several applications in the aerospace context require efficient, accurate, and robust simulations including complex geometries with moving bodies. For instance, the simulations addressing the flow over the blades of a rotorcraft, the drop of a cargo, or even the movement of control surfaces in a maneuver can be quite challenging due to the moving boundary interaction with a vortex dominated flow \cite{Rockwell1998, Abras2007, Costes2012}. The misrepresentation of these complex fluid structure interactions can affect multidisciplinary topics of an aerospace project. Eventually, when a vortex comes out of the tip of a blade, this vortex may travel up to important structures like the control surfaces and, then, may jeopardize some key engineering specifications of a project. Therefore, the consideration of movable structures with high accuracy and possibly long-time simulation turns out to be very desirable in these contexts.

In general, most of the Computational Fluid Dynamics simulations setup a unique mesh to represent the fluid-emerged body geometry altogether with a discrete specialized sample of the continuous physical domain into the computer. However, in scenarios with moving bodies, despite the existence of different methodologies to adapt the mesh given the body's movement \cite{Powell1993, Persson2005}, it can present a high computational cost during the simulation. Most commonly, a remeshing step is applied to the entire mesh domain everytime the deformed mesh reaches a low quality. In this sense, the treatment for moving boundaries uniquely described by single mesh is too cumbersome. On the other hand, research on overlapped multiple meshes has been addressed to overcome this limitation aiming to provide flexibility in the geometric representation, boundary movement and deformation, and high-order accuracy at a suitable cost \cite{Galbraith2013, Crabill2018, DuanThesis2019, Duan2020}. Commonly named as overset grids or Chimera mesh, the domain representation is setup by two mesh categories: a background mesh and a near-body mesh. The background mesh contains the overall simulation domain, while the near-body mesh moves with a fluid emerged body or fluid structure of interest such as a vortex. Thus, in the overset grid setup, only the near-body mesh needs to be adapted given a body boundary deformation or movement leaving a major part of the fluid domain in the background mesh as strictly static. 

In order to provide an overset grid solver, some additional procedures are often necessary to couple the solution data from one mesh to the other. At the preprocessing stage, a connectivity relationship needs to be established so that the cells at the external boundary of the near-body mesh can identify donors cells from the background. Furthermore, some background mesh cells are fully-overlapped by the near-body mesh and then no solver iteration is needed, once the simulation can be provided by the near-body mesh cells in these areas. The background donor cells will then interface skipped cells where the solution is not provided and, in a similar manner, the receptor-donor relationship is defined to gather data from the near-body mesh. Moreover, the data communication is done at each time integration step and applied as weak boundary condition at the overset region. 

Nonetheless, previous benchmark overset grid solvers only considered 2nd order resolution for the near-body mesh and linear interpolations for the data communication procedure \cite{Suhs2003, Sankaran2011}. With a low-order solver, not only the physical representation can be affected through a low resolution of phenomena such as vortex wakes and strong discontinuities, but also the computational cost is higher given a target order of accuracy which can be cumbersome and only achievable over mesh refinement. On the other hand, recent spatial discretization methods have been proposed providing arbitrary efficiency, robustness and accuracy in the high-order community \cite{SD_I:06, SD_II:07, SD_May:06, KrisVanDenAbeele2009, Moreira2016}. 

In the recent decade, a high-order method framework, named as FR/CPR for Flux Reconstruction/Correction Procedure via Reconstruction, has presented state-of-the-art solutions for several benchmark validation tests, in a broad and efficient manner, consolidating several high-order methods in a concise formulation \cite{Gao2013}. Due to the aforementioned complexities to provide a proper accurate and efficient solution on unsteady aerodynamics applications, a high-order method becomes important and a suitable candidate for the present work. Additionally, high-order meshes, which better represent complex geometries due to curvature representation from its curved cells, have been reported to indeed impact in the achievement of the high-order method accuracy \cite{Andre2018} and, therefore, they are also relevant for this project.

\section{Objective}
The present work aims to study approaches to achieve a high-order accurate and conservative interpolation between unstructured overset grids. The order of accuracy of the implementation is investigated using the common high-order benchmark test cases for both single and overset grid simulations. Since low-order meshes can limit the accuracy of a high-order method by producing non-physical oscillations over unsuitable geometrical boundary representations, curved meshes are also considered for the present high-order solver. Finally, the work discusses about some challenges in the overset region communication when considering curved cells.

\section{Dissertation Outline}
The sections of this dissertation to address the topics of the present work are organized as follows. Section 2 describes the physical formulation used throughout the project. Section 3 reviews the numerical method based on a high-order polynomial representation coupled with high-order meshes. Section 4 contains a detailed discussion for the overset grid methodology including the geometric search procedure and conservative high-order interpolation. Section 5 presents the results comparing single and overset grid and high-order accuracy test cases. Finally, concluding remarks and future work extensions for the present project are discussed in Section 6.