\section{Concluding Remarks}

In the present work, a high-order solver coupled with a conservative high-order face-based interpolation approach for overset grids is implemented and its main aspects are addressed. The physical phenomena are represented by the 2-D Euler equations numerically discretized with an arbitrarily high-order polynomial-based Spectral Difference method. High-order meshes are considered in the solver and used to improve the receptor-donor connectivity by leveraging high-order nodes at the curved cell interfaces. A 3rd-order Strong-Stability-Preserving Runge-Kutta method with 3 stages is used as explicit time-marching approach in the present effort, showing satisfactory residue convergence.

The solver has been successfully validated through the tests for both low and high order achieving the expected results, as indicated in the literature. The overset results obtained have demonstrated the capability of the kd-tree for the geometric searching problem as well as the face-based interpolation approach as candidate to preserve the solution order of accuracy and conservation. The mesh generation for the overset cases has been demonstrated as a crucial procedure to achieve the presented results, once several edge cases can appear around the overset region. In particular, for high-order curved meshes, not all of the inward fringe-circuit flux-points lay inside of the near-body mesh domain. Therefore, the flux can not be reconstructed in these scenarios and an alternative additional step is proposed to overcome this issue, adding robustness to the presented solver by reactivating specific hole cells at the fringe-circuit.

Furthermore, the methodology for the high-order overset grid solver presented satisfactory results over the inviscid cylinder and isentropic vortex test cases. In the cylinder case, the expected double symmetry solution, for instance, for the Mach number contours, around the cylinder wall is achieved when the cylinder curvature is represented by the high-order mesh. Moreover, for the isentropic vortex, not only the presented solution fully transported the vortex from the near-body mesh to the background mesh, but also effectively preserved the vortex symmetry and magnitude with an adequate order of magnitude for the L2-norm entropy error, when compared to the literature. 

The CPU performance tests show satisfactory results comparing the single and overset vortex cases presenting similar median values of CPU time per iteration for different solver orders and mesh sizes. The reason is due to the asynchronously calculation of the residue over the overset meshes implemented in the present code which overcomes the additional time cost of the overset steps.
\\
\\
\\
\\


\section{Future Work}
In Unsteady Aerodynamics applications, several numerical experiments consider moving boundaries to address the physical representation of movable objects inside the fluid domain. The present work discusses and presents results over a state-of-the-art methodology for overset grids with high-order conservative interpolation applied to some validation controlled applications. Apart for the initially intended investigation, additional aspects can be addressed for future research. 

As demonstrated throughout the present effort, the overset grid conservative interpolation methodology has shown interesting results for unstructured and curved meshes. In this scenario, difficulties can arise in the imposition of periodic boundary conditions depending on the outer boundary geometry in unstructured grids. In general, a periodic boundary is forced to establish a 1:1 cell and interface point relation between two far apart boundaries. The implementation of this type of boundary condition can bring considerable complexity for unstructured grid solvers. On the other hand, there are several applications in which the use of periodic boundary conditions would be extremely useful, especially for reducing the computational requirements. Nonetheless, an overset grid coupled with a conservative high-order interpolation can be a suitable candidate for the periodic boundary condition due to its flexibility and accurate data communication.

Furthermore, the capability to provide a suitable representation of strong discontinuities in the simulation is extremely relevant for several aerospace applications. Commonly, the use of limiter formulation addresses local test markers to identify discontinuities in the simulation. For some limiters, the information around the vicinity of a cell is necessary to build the local test, which at an overset region can be challenging. Therefore, a representation of discontinuities could be a possible extension of the use of the present methodology by investigating how to couple limiter formulations with the present data communication approach.